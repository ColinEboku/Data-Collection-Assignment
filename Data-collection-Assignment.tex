
\documentclass{article}                    % article class

\begin{document}
\begin{titlepage} % Suppresses headers and footers on the title page

	\centering % Centre everything on the title page
	
	\scshape % Use small caps for all text on the title page
	
	\vspace*{\baselineskip} % White space at the top of the page
	
	%------------------------------------------------
	%	Title
	%------------------------------------------------
	
		
	\vspace{0.80\baselineskip} % Whitespace above the title
	
	{\LARGE MAKERERE UNIVERSITY\\ COLLEGE OF COMPUTING AND INFORMATION SCIENCES\\ DEPARTMENT OF COMPUTER SCIENCE\\BIT2207: RESEARCH METHODOLOGY\\LECTURER: MR ERNEST MWEBAZE\\} % Title
	
	\vspace{8.00\baselineskip} % Whitespace below the title
	

	
	%------------------------------------------------
	%	Student
	%------------------------------------------------
	
	\vspace{0.5\baselineskip} % Whitespace before the student details
	
	{\scshape\Large NAME:EBOKU EYAYU COLIN\\REG. NO:16/U/4686/PS\\STUDENT NO.216002573 \\} %student details
	
	\vspace{0.5\baselineskip} % Whitespace below the student details
\end{titlepage}
\newpage
{\textbf{DRIVING HABITS IN UGANDA}}
\section{Abstract}
In Uganda, owning a car is a sign of affluence and it is expected that only the rich can afford to drive. However, driving in Uganda has alot of technicalities involved.
\section{Introduction:  }
Driving in Uganda can be an enjoyable experience or a pain in the neck. The habits of drivers in Uganda are unhealthy at best. Here we try to analyse the way a Ugandan driver(including cyclist also known as boda-bodas) thinks and what they base on to make decisions while on the road. 
A holder of a valid driver license issued by his country of residence is permitted to drive in Uganda for three months if the driver license is in English. After that period each driver will be required to obtain a Ugandan driving license. If a driver license is not in English, then a driver should have an approved notarized translation of his driver license.
\section{Literature Review}
A literature review was carried out to find out other people's experiences driving in Uganda.
Mark Penhallow writes that some rules of engagement do exist (such as driving on the left), but these appear to be advisory only and are frequently ignored. Consequently, there is no point of waving a “Highway Code” at anyone or claiming any self-righteousness, such as “but it’s my right of way!” or “I got here first” as this merely produces looks of utter perplexion and shrugged shoulders.
\section{Problem statement  }
The habits of drivers in Uganda are a cause of many traffic accidents and traffic jams in the country and can be solved by identifying the various habits and educating the drivers on how to use the roads safely.
\section{Scope}
This research is aimed at identifying:
\vspace{1.08\baselineskip}
{\newline}
1.The various habits and actions undertaken by Ugandan drivers that are deemed dangerous
\vspace{1.08\baselineskip}
{\newline}
2. The group of drivers that are notorious for reckless driving in the country and their particular misdemeanours.
\vspace{1.08\baselineskip}
{\newline}
3.Factors that encourage the current stae of affairs on teh Ugandan roads
\subsection{Objectives}
1. Developing strategies to increase road safety in the country.
\vspace{1.08\baselineskip}
{\newline}
2. Equiping the traffic police with skills to curb these malpractices 
\vspace{1.08\baselineskip}
{\newline}
3. Developing a system to monitor the roads in the country and identify the perpetrators of the law
\vspace{1.08\baselineskip}
{\newline}
4. Devising strategies to discourage drivers from engaging in dangerous driving habits
\section{Methodology}
A systematic literature search was conducted, including empirical, theoretical and methodological research on habits related to road safety. 
\vspace{2.00\baselineskip}
\subsection{Results}
It was found that driving habits are established as a result of individual dispositions as well as social norms and cultural values. A general scheme for categorising and operationalizing driving styles was suggested. On this basis, existing literature on driving habits and indicators was reviewed. Links between driving habits and road safety were identified and individual and socio-cultural factors influencing driving style were reviewed.

Additionally,the bigger the vehicle, the better. The presence of “bull bars” on the front of your vehicle helps intimidate other road users; show them you mean business! Loud horns are a norm, together with full-beam lights for after dark.
The presence of other miscellaneous items such as indicators, tyre treads, brakes and windscreen wipers is usually an indication that the vehicle belongs to an affluent foreigner or even richer NGO.
Flash of lights means “go ahead” or equally the opposite “I’m coming through.”
A right hand indicator means either “I’m pulling off to the left, so you can overtake me” or occasionally, “I’m going to turn right.”
You will notice that, yes, one meaning completely contradicts the other and could easily result in a collision.

Pedestrians are an occupational hazard of driving in Kampala, but can usually be safely ignored by drivers. Unfortunately, instead of remaining in the narrow areas of mud or dust that line Kampala’s roadsides, they have an inconsiderate tendency to walk in the roadway instead, often getting in the way of vehicles. A sharp blast from the car horn is usually sufficient to scatter them out of your way.

There exist pedestrians in white uniforms(alternatively reffered to as traffic police).
Unfortunately, these people are often quite fat (especially the successful ones), so are relatively easy to spot from afar. Their waving usually indicates that they are feeling hungry and want you to stop, so that they can tell you of some spectacularly imaginative reason why you should pay for their lunch. Consequently, they tend to be especially busy in the mornings, and less so after lunch. Their levels of activity also rise in the run-up to Christmas and when school fees are due. This in turn has led to diregard of the traffic rules in place already.
\section{Exisiting traffic rules in Uganda}
The main Ugandan road traffic rules include;
It is illegal to drive any motor vehicle without valid driver license.
The minimum age for driving is 18 years.
Drivers and all passengers are required to wear protective seat belts.
Motorcycle riders must wear protective helmets.
It is illegal to drive under the influence of drink or a drug. The permitted blood alcohol level is 0.08%.
The speed limit on highways is 100 km/h (62 mph), outside built up areas – 80 km/h (50 mph) and 50 km/h (31 mph) in urban built up areas.
The third-party insurance is mandatory.
To use of a mobile phone whilst driving is illegal.
A rear-facing baby seat must not be fitted into a seat protected by an active frontal airbag.
At intersections, drivers must give way to traffic from his right.
The police phone number is 999.

Additional information:
In case of an accident, a driver must give to a police officer his name and address, the name and address of the car owner, and the number of the registration plates of his vehicle. The driver must also report the accident in person at the police station as soon as possible, but not later than 24 hours after the accident.
If a driver needs to wear glasses or contact lenses, he must do that at all times while driving. The Uganda traffic police may require any driver to undertake an eyesight test.
If a driver is involved in an accident, he should immediately contact his insurer and take pictures of the scene of the accident and the damage of the car.
To rent a car, a driver must be at least 23 years old and has held a valid driver license for a period of not less than 2 years.
Any police officer in uniform may remove any vehicle from a metered parking place if it is not less than two hours from the time at which the vehicle first appeared to have been parked in contravention of the regulation.

However it is noted that these traffic rules are largely ignored.

\section{Examples of Data to be Collected}
The data that is collected fof this research includes;
Images of drivers diregarding traffic rules
recordings and audios of people describing their experience driving in Uganda
Locations notorious for having many reckless drivers.

\section{CONCLUSION:}
 Existing studies have addressed a wide variety of driving habits, and there is an acute need for a unifying conceptual framework in order to synthesize these results and make useful generalizations. There is a considerable potential for increasing road safety by means of behavior modification. Naturalistic driving observations represent particularly promising approaches to future research on driving styles.



\newpage
\section{Reference:}
1.http://journals.sagepub.com/doi/abs/10.1177/0018720815591313
\vspace{2.00\baselineskip}
{\newline}
2.http://www.adcidl.com/Driving-in-Uganda.html
\vspace{2.00\baselineskip}
{\newline}
3.https://www.carkibanda.com/en/blog/driving-on-uganda-roads-the-do-s-and-don-ts
\vspace{2.00\baselineskip}
{\newline}
4.http://muzungubloguganda.com/2012/11/driving-in-kampala/WpJxuehubIU
\vspace{2.00\baselineskip}
{\newline}
5.https://matadornetwork.com/notebook/driving-uganda/

\end{document}